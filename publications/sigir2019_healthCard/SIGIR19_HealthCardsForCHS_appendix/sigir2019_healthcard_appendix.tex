\documentclass[sigconf]{acmart}

\usepackage{booktabs} % For formal tables
\usepackage{mathtools}
\usepackage{hyperref}
\usepackage{listings}
\usepackage{subfigure}
\usepackage{color}
\usepackage{arydshln}
\usepackage{multirow}
\usepackage{pbox}
\usepackage{enumitem}


\DeclarePairedDelimiter\ceil{\lceil}{\rceil}
\DeclarePairedDelimiter\floor{\lfloor}{\rfloor}

%\fancyhead{}

%\settopmatter{printacmref=false, printfolios=false}

\definecolor{ForestGreen}{rgb}{0.13, 0.55, 0.13}
\newcommand{\revised}[1]{\textcolor{ForestGreen}{#1}}
\renewcommand{\thetable}{\Alph{table}}
% Copyright
\settopmatter{printacmref=false}
\setcopyright{none}
\renewcommand\footnotetextcopyrightpermission[1]{}
\pagestyle{plain}
%\setcopyright{acmcopyright}
%\setcopyright{acmlicensed}
%\setcopyright{rightsretained}
%\setcopyright{usgov}
%\setcopyright{usgovmixed}
%\setcopyright{cagov}
%\setcopyright{cagovmixed}


\newcommand\todo[1]{\textcolor{red}{#1}}
\newcommand\todel[1]{\textcolor{gray}{#1}}

%\newcommand{\todo}[1]{\textcolor{red}{#1}}
%\newcommand{\model}[1]{\textit{#1}}
\begin{document}

%\copyrightyear{2019} 
%\acmYear{2019} 
%\setcopyright{licensedothergov}
%\acmConference[SIGIR '19]{Proceedings of the 42nd International ACM SIGIR Conference on Research and Development in Information Retrieval}{July 21--25, 2019}{Paris, France}
%\acmBooktitle{Proceedings of the 42nd International ACM SIGIR Conference on Research and Development in Information Retrieval (SIGIR '19), July 21--25, 2019, Paris, France}
%\acmPrice{15.00}
%\acmDOI{10.1145/3331184.3331194}
%\acmISBN{978-1-4503-6172-9/19/07}
% Copyright
%\setcopyright{none}
%\setcopyright{acmcopyright}
%\setcopyright{acmlicensed}
%\setcopyright{rightsretained}
%\setcopyright{usgov}
%\setcopyright{usgovmixed}
%\setcopyright{cagov}
%\setcopyright{cagovmixed}

%\copyrightyear{2019} 
%\acmYear{2018} 
%\setcopyright{acmlicensed}
%\acmConference[SIGIR '19]{The 42nd International ACM SIGIR Conference on Research \& Development in Information Retrieval}{July 21--25, 2019}{Paris, France}
%\acmBooktitle{SIGIR '18: The 41st International ACM SIGIR Conference on Research \& Development in Information Retrieval, July 8--12, 2018, Ann Arbor, MI, USA}
%\acmPrice{15.00}
%\acmDOI{10.1145/3209978.3210088}
%\acmISBN{978-1-4503-5657-2/18/07}

\title[Health Card for CHS]{Appendix of Health Cards for Consumer Health Search}

%\affiliation{%
%  \institution{Queensland University of Technology}
%%  \streetaddress{P.O. Box 1212}
%  \city{Brisbane}
%  \state{Australia}
%%  \postcode{43017-6221}
%}
%\affiliation{%
%	\institution{University of Surabaya (UBAYA)}
%	%  \streetaddress{P.O. Box 1212}
%	\city{Surabaya}
%	\state{Indonesia}
%	%  \postcode{43017-6221}
%}
%\email{jimmy@hdr.qut.edu.au}

%\author{Jimmy}
%\affiliation{%
%  \institution{University of Queensland, Brisbane, Australia}
%}
%\affiliation{%
%	\institution{University of Surabaya (UBAYA), Surabaya, Indonesia}
%}
%\email{jimmy@uqconnect.edu.au}
%
%\author{Guido Zuccon}
%\affiliation{%
%  \institution{University of Queensland}
%  \city{Brisbane}
%  \state{Australia}
%}
%\email{g.zuccon@uq.edu.au}
%
%\author{Bevan Koopman}
%\affiliation{%
%	\institution{Australian E-Health Research Center, CSIRO}
%	\city{Brisbane}
%	\state{Australia}
%}
%\email{bevan.koopman@csiro.au}
%
%\author{Gianluca Demartini}
%\affiliation{%
%  \institution{University of Queensland}
%  \city{Brisbane}
%  \state{Australia}
%}
%\email{g.demartini@uq.edu.au}


% The default list of authors is too long for headers.
%\renewcommand{\shortauthors}{Jimmy, Zuccon, Demartini, \& Koopman}


%\settopmatter{printacmref=true}

%\input{1_Abstract.tex}
\maketitle
%
%%%%%%%%%%%%%%%%%%%%%%%%%%%%%%%%%%%% BEGIN BODY %%%%%%%%%%%%%%%%%%%%%%
%
%\input{2_introduction.tex}
%\input{3_RelatedWork.tex}
%\input{4_Experiments.tex}
%\input{5_Results.tex}
%%\input{6_Discussion.tex}
%\input{7_Conclusion.tex}




\bibliographystyle{ACM-Reference-Format}
\bibliography{health_card.bib} 

\appendix
\begin{table}[tb]
	\centering
	\small
	\caption{Perception questionnaire items, from Kelly et al.~\citep{Kelly2015}. Unless specified, options for each item ranged from 1 (very [neg]) to 5 (very [pos]). Where the [neg], [pos] labels were contextualised to the items, e.g. for item 1, [neg]=uninterested, [pos]=interested.} \label{table: pre questionnaire}
	\vspace{-10px}
	\begin{tabular}{p{8cm}}
		\toprule
		\textbf{Items}  \\
		\toprule
		
		\textbf{Interest \& knowledge} \\
		\begin{enumerate}[leftmargin=*,  topsep=0pt, labelsep=1pt]
			\item How interested are you to learn more about the topic of this scenario?
			\item How many times have you searched for information about the topic of this scenario? (1= never, 2= 1-2 times, 3= 3-4 times, 4= >5 times )
			\item How much do you know about the topic of this scenario? (1=nothing, 2=little, 3=some, 4=great deal)
		\end{enumerate} \\
		
		\toprule
		\textbf{Perceived Task Understandability} \\
		\begin{enumerate}[leftmargin=*,  topsep=0pt, labelsep=1pt]
			\setcounter{enumi}{4}
			\item How defined is the task in terms of the types of information needed to complete it?
			\item How defined is the task in terms of its expected solution?
		\end{enumerate} \\
		\bottomrule
	\end{tabular} 
\end{table}



\begin{table}[t]
	\centering
	\small
	\caption{User experience questionnaire items, adopted from Kelly et al.~\citep{Kelly2015}. Options for item 1 to 7 ranged from 1 (very [neg]) to 5 (very [pos]). Where the [neg], [pos] labels were contextualised to the items, e.g., for item 1, [neg]=difficult, [pos]=easy.} \label{table: user experience questionnaire}
	\vspace{-10pt}
	\begin{tabular}{p{8cm}}
		\toprule 
		\textbf{Items}  \\
		\toprule
		\textbf{Experienced Task Difficulty} \\
		\begin{enumerate}[leftmargin=*,  topsep=0pt, labelsep=1pt]
			\item How difficult was it to \textit{understand} the information the system presents?
			\item How difficult was it to determine when you \textit{have enough information} to finish the task?				  
			\item Overall, how \textit{difficult} was this task?
		\end{enumerate} \\
		
		\midrule
		\textbf{System Effectiveness Assessment} \\
		\begin{enumerate}[leftmargin=*,  topsep=0pt, labelsep=1pt]
			\setcounter{enumi}{4}
			\item Did the system provide \textit{sufficient information} to help me complete the task?
			\item How effective was the system in helping you to \textit{find the right information}?
		\end{enumerate} \\
		
		\midrule
		\textbf{Satisfaction \& Workload Assessment} \\
		\begin{enumerate}[leftmargin=*,  topsep=0pt, labelsep=1pt]
			\setcounter{enumi}{6}
			\item Overall, how \textit{satisfied} are you with your solution to this task?
			\item How would you describe the work you have done to complete this task? (1=very hard, 2=hard, 3=neutral, 4=easy, 5=very easy)
		\end{enumerate} \\
		\bottomrule%\hline 	
	\end{tabular} 
\end{table}



\begin{table}[tb]
	\centering
	\small
	\caption{Exit questionnaire. Options for each items 1 to 4 ranged from 1 (strongly disagree) to 5 (strongly agree).}\label{table: exit questionnaire}
	\vspace{-10pt}
	\begin{tabular}{p{8cm}}
		\toprule 
		\textbf{Items} \\
		\toprule
		\begin{enumerate}[leftmargin=*,  topsep=0pt, labelsep=1pt]
			\item The system was easy to use. 
			\item The system provided me with useful information.
			\item Overall, the quality of the results displayed by the system is similar to those I experienced in my everyday interaction with general-purpose search engines like Google and Bing.
			\item Overall, I am satisfied with the system.
			
			\item Did you notice the health card when completing the tasks for which the card was displayed? (Yes or No)
			\item Did you use the health card when completing the tasks for which the card was displayed? (Yes or No)
			
			\item If you have searched for health information before, have you seen a health card before? 
			\begin{itemize}
				\item I never used a search engine to search for health information before, 
				\item I never seen a health card before, or
				\item Yes, I have seen a health card before
			\end{itemize}
			
			\item If you have seen a health card before, based on your previous experience, you consider health card as being: 
			\begin{itemize}
				\item NOT relevant to my query or Relevant to my query
				\item Difficult to understand or Easy to understand
				\item NOT trustworthy or Trustworthy
			\end{itemize} 
		\end{enumerate} \\
		\bottomrule
	\end{tabular} 
\end{table}

\end{document}
